\chapter{Introduction}
\label{chap-one}

Traditionally, news and information about events would be available from few sources like television news channels and radios. With the upsurge of use of social networks over this decade, data is available on a real-time basis in humungous amounts on platforms like Twitter, Facebook and so on. To make true the statement that “information is wealth”, this overwhelming amounts of data has to be processed and converted into useful, precise information that provides important insights into different issues.

\section{Twitter}
Twitter is a social network where a user can post details or comments about a topic of his choice in 140 characters. Each such post is called a "tweet". These tweets also include URLs to different articles, hashtags that act as meta data about the tweet itself.  These tweets as micro-blogs where users can present any content in a precise manner. Popular tweets are tweeted again by other users and are called as “retweets”.Twitter has half a billion user accounts and the users send 175 million tweets every day. It is being used massively to post content on the web by a range of users ranging from politicians to pop stars to voice their opinions, support different causes or to just post what's on their mind. Twitter has become the strong social medium where users react to catastrophes like the Haiti earthquake or post about a power cut or express their constant emotions during a football match. A search on a popular event generates tens of hundreds of results per second. 

On the other side of the sphere, the incredible amounts of data generated by Twitter act as the source of data for many researchers these days. Twitter provides easy-to-use RESTful APIs to obtain tweets of any public user or to search for tweets and researchers are working on this data to generate useful information from them. They have been used in varied contexts to provide “summarized information”. Nichols et al have worked on generating a textual summary of soccer using the tweets from Twitter. \citet{DBLP:conf/icwsm/ChakrabartiP11} have presented the set of highlights or important sub events as a summary of structured, time-constrained events. \citet{Sakaki:2010:EST:1772690.1772777} detect occurrence and time of earthquakes or trajectory of typhoons from set of user tweets as summary. However, in the context of summarizing tweets of a single user, we feel that a small set of tweets that is representative of all the topics and events that a user tweets about can be considered as a summary. This  provides an idea of the users' interests and an insight into the kind of content the user usually tweets on Twitter.

\section{Goals}
In our thesis, we would like to accomplish the following goals
\begin{enumerate}
\item To study different algorithms used for summarizing micro blogs
\item To explore supervised and unsupervised techniques to summarize micro blogs
\item To test and analyze our findings with real users
\end{enumerate}

\section{Contribution}

Our work looks at summarizing the micro blogs of a single user.  We have looked at a new unexplored problem. We have conducted a pilot study to understand human perception about summarizing the tweets of a micro blog. We have reviewed existing approaches for summarizing micro blogs. We have explained that these approaches cannot be used to summarize small sets of micro blogs by a single user. We have explored supervised and unsupervised techniques to summarize micro blogs. We have presented the results and an analysis for each algorithm. We have described a modified version of KMeans which can be used to summarize tweets. This algorithm is flexible and can change the granularity in way users can understand it. We have evaluated our algorithm with real users and provided a comparative analysis with SUMMALLTEXT \cite{DBLP:conf/icwsm/ChakrabartiP11}. We have also provided a detailed statistical analysis of the comparative study.




