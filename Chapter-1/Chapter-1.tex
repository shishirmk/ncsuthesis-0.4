\chapter{Introduction}
\label{chap-one}

Traditionally, news and information about events would be available from few sources like television news channels and radios. With the upsurge of use of social networks over this decade, data is available on a real-time basis in humungous amounts on platforms like Twitter, Facebook and so on. To make true the statement that “information is wealth”, this overwhelming amounts of data has to be processed and converted into useful, precise information that provides important insights into different issues. In this chapter we discuss about micro blogs and reasons for summarizing micro blogs. 

\section{Twitter}
Twitter is a social network where a user can post details or comments about a topic of his choice in 140 characters. Each such post is called a "tweet". The word tweet will be used often in our thesis to refer to the posts of a micro blog. These tweets also include URLs to different articles, hashtags that act as meta data about the tweet itself.  One can mention a user by using a @ sign before the username in the post. We can reply to tweets, popular tweets are tweeted again by the followers of the original users and are called as “retweets”. A follower is a twitter user who has subscribed to receive your tweets. Twitter has half a billion user accounts around 140 million active monthly users and the users send 340 million tweets every day \cite{twitterstats}. It is being used massively to post content on the web by a range of users ranging from politicians to pop stars to voice their opinions, support different causes or to just post what's on their mind.

Twitter has become the strong social medium where users react to catastrophes like the Haiti earthquake, Fire Hazards, Twitter has been used as a loosely bound collaboration medium for people who are reacting to these natural calamities. Twitter has helped in organizing mass movements like the Arab spring. Sports like football, hockey and basketball fans tweet live updates about the game. They tweet there comments and opinions and emotions during the game. Twitter is a abundant source of information. 

An average user on twitter follows 126 people. Only 40\% of twitter users actually tweet frequently most twitter users dont tweet but they are still very active on twitter as they constantly check twitter for updates about the their favorite politicians, celebrities, sports teams, corporations etc. The famous people are content generators, they have numerous followers.  President Barack Obama has more than 14.5 million followers on twitter. Lady Gaga has more than 23.5 million followers on twitter. President Barack Obama's account sends out around 10-12 tweets per day. If one follows 126 such users, It could be estimated that on an average a twitter user has to check 1260 tweets per day. This requires significant amount of time. Often tweets about the same topic get repeated and paraphrased and a user has to waste time in going through such tweets. This time can be saved by recognizing the core topics of the tweets and summarizing the tweets before presenting it to the user. In general summarizing tweets of a user can help concisely express what the user is talking about.

On the other side of the sphere, the incredible amounts of data generated by Twitter act as the source of data for many researchers these days. Twitter provides easy-to-use RESTful APIs to obtain tweets of any public user or to search for tweets and researchers are working on this data to generate useful information from them. They have been used in varied contexts to provide “summarized information”. \citet{Nichols:2012:SSE:2166966.2166999} have worked on generating a textual summary of soccer using the tweets from Twitter. \citet{DBLP:conf/icwsm/ChakrabartiP11} have presented the set of highlights or important sub events as a summary of structured, time-constrained events. \citet{Sakaki:2010:EST:1772690.1772777} detect occurrence and time of earthquakes or trajectory of typhoons from set of user tweets as summary. However, in the context of summarizing tweets of a single user, we feel that a small set of tweets that is representative of all the topics and events that a user tweets about can be considered as a summary. This  provides an idea of the users' interests and an insight into the kind of content the user usually tweets on Twitter.

\section{Goals}
Summarizing tweets of a user has many uses. If the user tweets about the same topic often then its good to avoid repetition when presenting it to their followers. It can save time taken to go through all the different tweets a user gets in a day. Summarized tweets of a user can also be used to understand about the topics the user generally tweets about. This can assist in making decisions about following a user or not. For a user summarizing his own tweets and looking at them in a time line can give him a better idea about what he has tweeted about over a period of time and how his interests have changed. A summarization technique for small sets of tweets will help an average twitter user better cope with the information overload. 

In our thesis, we would like to accomplish the following goals
\begin{enumerate}
\item To study different algorithms used for summarizing micro blogs
\item To explore supervised and unsupervised techniques to summarize micro blogs
\item To test and analyze our findings with real users
\end{enumerate}

\section{Contribution}

Our work looks at summarizing the tweets of a single user.  We have looked at a new unexplored problem. We have conducted a pilot study to understand human perception about summarizing the tweets of a twitter user. We have reviewed existing approaches for summarizing tweets and micro blog posts in general. We have explained that these approaches cannot be used to summarize small sets of micro blogs by a single user. We have explored supervised and unsupervised techniques to summarize tweets. We have presented the results and an analysis for each algorithm. We have described a modified version of KMeans which can be used to summarize tweets. This algorithm is flexible and can change the granularity of the summary in a way users can understand and control it. We have evaluated our algorithm with real users and provided a comparative analysis with SUMMALLTEXT \cite{DBLP:conf/icwsm/ChakrabartiP11}. We have also provided a detailed statistical analysis of the comparative study.




